\documentclass[12pt,twoside]{article}
\usepackage{jmlda}
%\NOREVIEWERNOTES
\title
    [Задача поиска символов в текстах] % Краткое название; не нужно, если полное название влезает в~колонтитул
    {Задача поиска символов в текстах}
\author
    [Северилов~П.\,А.] % список авторов для колонтитула; не нужен, если основной список влезает в колонтитул
    {Северилов~П.\,А., Лемтюжникова~Д.\,В., Апишев~М.\,А.} % основной список авторов, выводимый в оглавление
    [Северилов~П.\,А.$^1$, Лемтюжникова~Д.\,В.$^1$, Апишев~М.\,А.$^2$] % список авторов, выводимый в заголовок; не нужен, если он не отличается от основного
\thanks
    {
    	%Работа выполнена при финансовой поддержке РФФИ, проект \No\,00-00-00000. Научный руководитель:  
   Задачу поставил:  Лемтюжникова~Д.\,В., 
    Консультант: Апишев~М.\,А.}
\email
    {severilov.pa@phystech.edu, daratigra@icloud.com, great-mel@yandex.ru}
\organization
    {$^1$Московский физико-технический институт (МФТИ); 
    	
	 $^2$Московский государственный университет имени М.В.Ломоносова (МГУ им. М.В.Ломоносова)}
\abstract
    {В работе рассматривается задача поиска символов в тексте. Определение в тексте средств выразительности  таких, как метафоры, аллегории и пр., у экспертов происходит в ручном режиме, и процесс никак не был автоматизирован. В простейшем случае эта задача сводится к проблеме Sequence Labeling на размеченной выборке. Мы сравниваем работы нескольких современных алгоритмов решения Sequence Labeling и определяем применимость данных методов к нашей задаче.  Также предложена метрика качества классификатора для определения символ/не символ.

\bigskip
\textbf{Ключевые слова}: \emph {распознавание символов, LSTM нейронные сети, sequence labeling, cкрытые марковские модели}.}
\titleEng
    {JMLDA paper example: file jmlda-example.tex}
\authorEng
    {Author~F.\,S.$^1$, CoAuthor~F.\,S.$^2$, Name~F.\,S.$^2$}
\organizationEng
    {$^1$Organization; $^2$Organization}
\abstractEng
    {This document is an example of paper prepared with \LaTeXe\
    typesetting system and style file \texttt{jmlda.sty}.

    \bigskip
    \textbf{Keywords}: \emph{keyword, keyword, more keywords}.}
\begin{document}
\maketitle
%\linenumbers
\section{Введение}
Современные модели для обработки текстов не справляются с главной особенностью языка — неоднозначностью смысла высказывания. Текст воспринимается ими буквально, и различные средства художественной выразительности, в частности, символы, метафоры, аллегории и др. никак не интерпретируются в правильном смысле. Так, например, выражение «золотые руки» вероятнее всего будет понято моделью, как «руки из золота» вместо верного «умения очень хорошо делать что-либо». Автоматизация поиска выражений с неоднозначным смыслом продвинет обработку текстов на более глубокий уровень понимания текста у моделей.

Данная задача поиска символов в тексте может быть сведена к Sequence Labeling -- более общей задаче, широко распространенной в NLP. Как правило, рассматриваются три конкретных типа sequence labeling: тегирование частей речи (part-of-speech tagging), распознавание именованных сущностей (named entity recognition) и синтаксический анализ (shallow parsing). Данные методы вполне подходят к нашей задаче.

Традиционно задачи sequence labeling решаются с использованием линейных статистических моделей, например: скрытые марковские модели (HMM),
марковские случайные поля (CRFs). Реализация решений происходит с помощью различных архитектур нейронных сетей. В данной статье мы сравниваем результаты работы нескольких state-of-the-art архитектур для sequence labeling применительно к задаче поиска символов. 

Архитектуры таких нейронных сетей главным образом базируются на рекуррентных сетях, а конкретнее Bidirectional LSTM с использованием CRF. Рассматриваем три подхода. Первый -- это классическая реализация BiLSTM. Затем аналогичная архитектура с изменением в том, что на вход сети будут подаваться не векторные представления слов в целом, а каждый символ по отдельности. И последняя -- основанная на идее предыдущей модели, но с добавлением механизма внимания (attention) в архитектуру (Transformer).

Для того, чтобы определить скрытый смысл высказывания, для начала нужно научиться понимать, а есть ли он вообще в конкретном выражении. Поэтому мы решаем задачу классификации символ/не символ. Тестирование проводится на тексте романа М.А.Булгакова "Мастер и Маргарита". Основная сложность заключается в получении достаточного объёма обучающих данных, то есть требуется по имеющейся небольшой экспертной разметке получить выборку большего размера. Наличие разметки позволит провести эксперименты с подбором оптимальной модели и в целом определить применимость данных методов для решения задачи поиска символов.
%\cite{author09anyscience,myHandbook,author09first-word-of-the-title,voron06latex,author-and-co2007,Lvovsky03}.

\section{Название раздела}
smth

\paragraph{Название параграфа.}
%Первый раздел может содержать формальную постановку задачи,
%основные определения и~обозначения,
%известные факты, необходимые для понимания основных результатов работы,
%и~т.\,п.
another smth

\paragraph{Теоретическая часть} 


\section{Заключение}
final


\bibliographystyle{unsrt}
\bibliography{jmlda-bib}
%\begin{thebibliography}{1}

%\bibitem{author09anyscience}
%    \BibAuthor{Author\;N.}
%    \BibTitle{Paper title}~//
%    \BibJournal{10-th Int'l. Conf. on Anyscience}, 2009.  Vol.\,11, No.\,1.  Pp.\,111--122.
%\bibitem{myHandbook}
%    \BibAuthor{Автор\;И.\,О.}
%    Название книги.
%    Город: Издательство, 2009. 314~с.
%\bibitem{author09first-word-of-the-title}
%    \BibAuthor{Автор\;И.\,О.}
%    \BibTitle{Название статьи}~//
%    \BibJournal{Название конференции или сборника},
%    Город:~Изд-во, 2009.  С.\,5--6.
%\bibitem{author-and-co2007}
%    \BibAuthor{Автор\;И.\,О., Соавтор\;И.\,О.}
%    \BibTitle{Название статьи}~//
%    \BibJournal{Название журнала}. 2007. Т.\,38, \No\,5. С.\,54--62.
%\bibitem{bibUsefulUrl}
%    \BibUrl{www.site.ru}~---
%    Название сайта.  2007.
%\bibitem{voron06latex}
%    \BibAuthor{Воронцов~К.\,В.}
%    \LaTeXe\ в~примерах.
%    2006.
%    \BibUrl{http://www.ccas.ru/voron/latex.html}.
%\bibitem{Lvovsky03}
%    \BibAuthor{Львовский~С.\,М.} Набор и вёрстка в пакете~\LaTeX.
%    3-е издание.
%    Москва:~МЦHМО, 2003.  448~с.
%\end{thebibliography}

% Решение Программного Комитета:
%\ACCEPTNOTE
%\AMENDNOTE
%\REJECTNOTE
\end{document}
